\documentclass[12pt,a4paper,twoside]{report}
\usepackage[utf8]{inputenc}
\usepackage[T1]{fontenc}
\usepackage{polski}
\usepackage{amssymb}
\usepackage[polish]{babel}


\usepackage{standalone}
\usepackage{amsfonts}
\usepackage[left=2.5cm,right=2.5cm,top=2.5cm,bottom=2.5cm]{geometry}

\usepackage{import}
\usepackage{indentfirst}
\usepackage{url}
\usepackage[labelsep=period, font=bf]{caption}

\usepackage{fancyhdr}
\usepackage{longtable}
\usepackage{url}
\usepackage{float}
\usepackage{pdfpages}
\usepackage{indentfirst}
\usepackage{fancyhdr}
\usepackage{listings}
\lstset{basicstyle=\small\sffamily, captionpos=b, breaklines=true}
\usepackage{courier}
\usepackage{pdfpages}

\linespread{1.3}
\hyphenpenalty=3000
\pretolerance=3000
\tolerance=5000 
\emergencystretch=10pt
\setlength{\headheight}{15pt}
\widowpenalty=10000
\clubpenalty=10000

\usepackage{titlesec}

\titleformat{\chapter}{\bf\huge}{\thechapter.}{20pt}{\huge\bf}

\usepackage[acronym,toc,shortcuts,nopostdot]{glossaries}

\usepackage{csquotes}
\DeclareQuoteAlias{croatian}{polish}
\usepackage{textcomp}
\DeclareUnicodeCharacter{00A0}{~}
\makeglossaries
\loadglsentries{rozdzialy/akronimy}

\usepackage{hyperref}
\hypersetup{
    colorlinks,
    citecolor=black,
    filecolor=black,
    linkcolor=black,
    urlcolor=black
}

\usepackage[
  backend=biber,
  style=numeric
]{biblatex}
\addbibresource{./rozdzialy/bibtex.bib}
\usepackage{blindtext}

\author{No Ty}
\title{Tytul pracy tu}

\makeatletter
\renewcommand{\maketitle}{\begin{titlepage}
    \vspace*{1cm}
    \begin{center}
    	Politechnika Poznańska\\
    	Wydział Elektryczny\\
    \end{center}
    \vspace{3cm}

    \begin{center}
     \LARGE\textbf{\textsc{\@title}}\\
      \Large\textsc{\@author}
    \end{center}
    \vspace{0.5cm}
    \begin{flushright}

    \vspace{9cm}
     {\small Praca inżynierska
napisana pod kierownictwem}\\
         tego tamtego 
     \end{flushright}
    \vspace*{\stretch{6}}
    \begin{center}
    Poznań, 2017
    \end{center}
  \end{titlepage}%
}
\makeatother


\usepackage{enumitem}
\setitemize[1]{label=$\bullet$}
\usepackage{amsmath}
\usepackage{amsfonts}
\usepackage{amssymb}
\usepackage{amsthm}
%\setlength{\footskip}{10mm}
\pagenumbering{arabic}

\begin{document}
	\maketitle

  %Dedykacja 
      \newpage
    \begin{minipage}{6cm} \hfill
  \end{minipage}  
  \begin{minipage}{9cm}
    \vspace{18.5cm}
    {\fontfamily{cmss} \fontshape{it} \selectfont Tutaj możecie napisać komu dziękujecie za to że wam się udało i w ogóle}
  \thispagestyle{empty}
  \end{minipage}
  %Koniec dedykacji
  %Streszczenie po PL
  \begin{abstract}
    Trochę o celu pracy, streszczenie rozdziałów i jakieś podsumowanie z zakończenia
  \end{abstract}

  %Abstract in EN
  \renewcommand{\abstractname}{Abstract}
  \begin{abstract}
  I AM POTATO 
  \end{abstract}
	\tableofcontents
	\pagestyle{plain}
	\chapter*{Wstęp}
	\addcontentsline{toc}{chapter}{Wstęp}  
		\import{rozdzialy/}{1}
	\chapter{1}
		\import{rozdzialy/}{2}
	\chapter{2}
		\import{rozdzialy/}{3}
	\chapter{3}
		\import{rozdzialy/}{4}
	\chapter{4}
		%\import{rozdzialy/}{5}
	\chapter{5}
		%\import{rozdzialy/}{6}
	\chapter{6}
	    %\import{rozdzialy/}{7}
	\chapter*{Zakończenie}
  %\addcontentsline{toc}{chapter}{Zakończenie  
		%\import{rozdzialy/}{zakonczenie}
	\printglossary[type=\acronymtype,style=long,title=Lista akronimów]
  \import{rozdzialy/}{literatura}
\end{document}\grid
